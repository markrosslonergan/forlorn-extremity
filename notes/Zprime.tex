%SAVEDAT= 1443708096
\documentclass[11pt, a4paper]{article} 
\usepackage[force]{feynmp-auto}
\usepackage{jheppub,multirow,relsize,slashed}

\newcommand{\marrow}[5]{%
    \fmfcmd{style_def marrow#1
    expr p = drawarrow subpath (1/4, 3/4) of p shifted 6 #2 withpen pencircle scaled 0.4;
    label.#3(btex #4 etex, point 0.5 of p shifted 6 #2);
    enddef;}
    \fmf{marrow#1,tension=0}{#5}}

\newcommand{\refeq}[1]{Eq.~(\ref{#1})}
\newcommand{\refeqs}[2]{Eqs.~(\ref{#1})~and~(\ref{#2})}
\newcommand{\refeqss}[3]{Eqs.~(\ref{#1}), (\ref{#2})~and~(\ref{#3})}
\newcommand{\reffig}[1]{Fig.~\ref{#1}}
\newcommand{\reffigs}[2]{Figs.~\ref{#1}~and~\ref{#2}}
\newcommand{\refsec}[1]{Section~\ref{#1}}
\newcommand{\refapp}[1]{Appendix~\ref{#1}}
\newcommand{\reftab}[1]{Table~\ref{#1}}
\newcommand{\refref}[1]{Ref.~\cite{#1}}
\newcommand{\refrefs}[2]{Refs.~\cite{#1}~and~\cite{#2}}

\def\abelian{abelian}
\def\nonabelian{non-abelian}
\def\lagrangian{lagrangian}
\def\eg{\emph{e.g.}}
\def\ie{\emph{i.e.}}
\def\aka{\emph{a.k.a.}}

\newcommand{\Dslash}{\ensuremath D\hspace{-0.24cm}\raisebox{1pt}{/}\hspace{0.05cm}}
\newcommand{\Xslash}{\ensuremath X\hspace{-0.26cm}\raisebox{1pt}{/}\hspace{0.06cm}}
\newcommand{\Wslash}{\ensuremath W\hspace{-0.3cm}\raisebox{1pt}{/}\hspace{0.06cm}}
\newcommand{\dslash}{\ensuremath \partial\hspace{-0.18cm}\raisebox{1pt}{/}\hspace{0.04cm}}
\newcommand{\gp}{\ensuremath g^\prime}

\newcommand{\lorem}{ \textcolor[rgb]{0.8,0.8,0.8}{Lorem ipsum dolor sit amet, consectetur
adipiscing elit, sed do eiusmod tempor incididunt ut labore et dolore magna
aliqua. Ut enim ad minim veniam, quis nostrud exercitation ullamco laboris nisi
ut aliquip ex ea commodo consequat. Duis aute irure dolor in reprehenderit in
voluptate velit esse cillum dolore eu fugiat nulla pariatur. Excepteur sint
occaecat cupidatat non proident, sunt in culpa qui officia deserunt mollit anim
id est laborum.}}

%%%%%%% A few editorial macros. %%%%%%%

\definecolor{light-gray}{gray}{0.65}
\newcommand{\filler}[1]{{\color{light-gray}{#1}}}

\newcounter{CommentCount}
\setcounter{CommentCount}{1}

\newcommand{\marcom}[2]{\textsuperscript{\textcolor{#1}{\theCommentCount}}\marginpar{\textsuperscript{\textcolor{#1}{\theCommentCount}}\textcolor{#1}{{\small#1: #2}}}\stepcounter{CommentCount}}

\newcommand{\newtext}[2]{\textcolor{#1}{\ul{#2}}}

% Add your own colour down here... 
\definecolor{PB}{rgb}{0.9,0,0}

%%%%%%%%%%%%%%%%%%%%%%%%%%%%%%%%%%%%%%%


\title{Temporary title}

\author{One}
\author{Two}
\author{Three}

\affiliation{Institute for Particle Physics Phenomenology, Department of
Physics, Durham University, South Road, Durham DH1 3LE, United Kingdom}

\emailAdd{one@durham.ac.uk}
\emailAdd{two@durham.ac.uk}
\emailAdd{three@durham.ac.uk}

\abstract{~}

\begin{document} 

%\maketitle

\section{$N$ decay rates}

The sterile neutrino has couplings to the $Z$ and $W^\pm$ bosons as well as the
new $Z^\prime$. This means that its decay rates are a combination of the
standard heavy sterile weak decay rates (pure leptonic and semileptonic rates),
combined with novel processes driven by the $Z\prime$. 

%For relatively light
%$N$, we only expect corrections to the rates of processess like $N\to
%\nu\nu\nu$ and $N \to \nu e^+e^-$. However, we \emph{expect} the three neutrino
%rate to be very suppressed ($Z^\prime$-$\nu_i\nu_j$ vertices are quite
%suppressed by mixing) while the electron rate will be much larger than its
%standard model counterpart. In what follows we compute the $Z^\prime$ mediated
%decay rate: this is a necessary input to our model in itself, but it will also
%allow us to check in which regions of parameter space we can neglect the $Z$
%mediated contributions to the same process.
%
%As an aside from compiling this document, it is interesting to note that in our
%model the $Z^\prime$ produces no FCNCs, aside from in the neutrinos, does not
%lead to NSIs in electrically neutral matter (as it couples to non-neutrinos
%proportionally to their electric charge), and has a significantly suppressed
%coupling to pseudoscalar mesons (due to its vectorial nature). 
%
%
%
%\section{Standard and non-standard rates}
%
In our model, the $W$-vertices are unchanged (they are still only modified by
the PMNS matrix), and so at tree-level we can use the usual expressions for the
decay rates of any process with only CC diagrams. However, the $Z$-boson mediated 
decays should be recomputed. Not only does the $Z$-boson have
modified couplings in our model (and crucially slightly modified axial-vector
relationships), but for any process with a $Z$ exchange, we could also exchange
the $Z^\prime$. These diagrams should be allowed to interfere. 

We might reasonably expect such interference to be small. Especially when
$Z^\prime$ mediation produces a significantly enhanced decay rate. However, 
when pushing the $\chi^2$ value down, this should be borne in mind.

\subsection{$N\to\nu\nu\nu$ (TO DO)}

This is normally the dominant decay rate below the $\pi$ mass threshold.  It
has a very similar structure to the $e^+e^-$ decay computed below, but with
different coupling constants.  Specifically, the coupling of $Z^\prime$ to two
light neutrinos is doubly suppressed by the small sterile mixing angles. It is
expected to be negligibly enhanced by the $Z^\prime$. 

One of the complications in computing this decay rate is the various
permutations of three mass eigenstates which are allowed for majorana
particles. This is more incentive not to do it.


\subsection{$N\to\nu\pi^0$}

This is purely neutral current decay, and would expect diagrams involving an
exchanged $Z^\prime$.
%
Their effect is expected to be small, because of the predominately vectorial
couplings of the boson to fermions. At first order in $\chi$, there is actually
no coupling of our boson to pseudoscalar mesons. The leading sensitivity then
comes from the modifed couplings to the $Z$ boson. However, the coupling
between pion and $Z^\prime$ enters with finite $Z^\prime$ mass at first order
in $\chi$, or at third order in $\chi$. As it's quick to do, we recompute this
rate with both $Z$ and $Z^\prime$ bosons, ensuring these effects are taken into
account.

\subsubsection{Recap of basic calculation}

We define our interaction vertices as in \refapp{app:vertices}.
%
Because we are dealing with a bound state, we must make use of the pion decay constant, and start from the transition matrix element. We consider the process with only the $Z$ mediator to start:
%
\begin{align*}   \left\langle k_1, k_2\right| iT \left|p\right\rangle &= i\mathcal{M}(2\pi)^4\delta^4(p-k_1-k_2), \\
%
&= \left.\left\langle k_1,k_2\right|T\left\{e^{i\int d^4x \mathcal{L}_\text{Int}} \right\} \left|p\right\rangle\right|_\text{conn. amput.}\\
%
&= \frac{ic^u_Ac^{4i}}{2}\left(\frac{g}{c_W}\right)^2 \frac{\overline{u}(k_1) \gamma^\mu P_\text{L} u(p)}{k_2^2-M_Z^2}\int d^4x e^{i(k_1-p)\cdot x} \left\langle k_2\right| \left(\overline{u}\gamma^\mu\gamma^5u - \overline{d}\gamma^\mu\gamma^5d\right)\left|0\right\rangle 
%
\end{align*}
%
We cannot compute the final amplitude, but use the pion decay constant in the form  
%
\[ \int d^4x e^{i(k_1-p)\cdot x}  \left\langle k_2\right| \left(\overline{u}\gamma^\mu\gamma^5u - \overline{d}\gamma^\mu\gamma^5d\right)\left|0\right\rangle = if_\pi k_2^\mu (2\pi)^4\delta^4(p-k_1-k_2).  \]
%
All together we find, 
%
\begin{align*}
%
i\mathcal{M} &= \left(\frac{g}{c_W}\right)^2\frac{c^u_A c^{4i} f_\pi }{2} k_2^\mu\frac{\overline{u}(k_1) \gamma^\mu P_\text{L} u(p)}{k_2^2-M_Z^2}
%
\end{align*}

\subsubsection{Full interference calculation}

As the matrix element for a single mediator is so simple, we can also include the decays via
$Z$-boson and their interference. The full matrix element is given by the sum of two terms like the one calculated above
%
\begin{align*}
%
i\mathcal{M} &= \frac{1}{2}\left(\frac{g}{c_W}\right)^2\left(\frac{c^u_A c^{4i}}{k_2^2-M_Z^2} + \frac{d^u_A d^{4i} }{k_2^2-\mu^2} \right) f_\pi k_2^\mu \overline{u}(k_1) \gamma_\mu P_\text{L} u(p),\\
%
&\equiv F_i f_\pi k_2^\mu 
\overline{u}(k_1) \gamma_\mu P_\text{L} u(p). 
%
\end{align*}
%
Where $F_i$ is the product of the two initital factors. Although $F_i$ depends on $k_2^2$, when we impose energy conservation this is just a function of particle masses as $k_2^2=m_{\pi^0}^2$, so this term is just an overall scaling of the rate. Completing the calculation leads us to 
%
\begin{align*}
%
\Gamma_i = \frac{1}{4\pi}\left|F_i\right|^2 f^2_\pi m_\pi m_\mu^2\left(1-\frac{m^2_{\pi^0}}{m^2_s}\right)^2.
%
\end{align*}
%
This should be summed over the three final mass states $\Gamma = \sum_{i=1}^3 \Gamma_i$. We can do this by hand, by defining a simplified set of couplings,
%
\[   c^{4i} \equiv U_{s4}^*U_{si}\widetilde{c}^4\qquad \text{and}\qquad  d^{4i} \equiv U_{s4}^*U_{si}\widetilde{d}^4, \]
%
which allows us to factorize out the dependence on $i$, 
%
\[ \sum_{i=1}^3\left| F_i\right|^2 = |U_{s4}|^2\left(1-|U_{s4}|^2\right) \frac{G_\text{F}^2}{2} \rho. \]
%
where we define a new parameter 
%
\[  \rho = 16M_Z^4 \left |\frac{c^u_A \widetilde{c}^{4}}{k_2^2-M_Z^2} + \frac{d^u_A \widetilde{d}^{4} }{k_2^2-\mu^2} \right|^2. \] 
%
This gives us the final formula
%
\[  \Gamma =  \left(1-\sum_{\alpha \neq s}  |U_{\alpha 4}|^2\right) \sum_{\alpha \neq s} |U_{\alpha 4}|^2 \rho \frac{1}{8\pi} G_F^2 f^2_\pi m_\pi m_\mu^2\left(1-\frac{m^2_{\pi^0}}{m^2_s}\right)^2. \]

\subsubsection{Sanity check}

As a check we set $\chi=0$, which means $d^{ij}=d^u_A=0$ and the $c$ coupings
become their $\nu$SM values, $c^u_A = \widetilde{c}^4 = \frac{1}{2}$.
%
The $\rho$ parameter can be shown, expanding to first order in in $m_\pi/M_Z$, to take the value $1$,%
%
%We find the factor $F_i$ becomes 
%%
%\[  |F_i|^2 = \left| \frac{1}{8} \left(\frac{g}{c_W}\right)^2 \frac{U^*_{s4}U_{si}}{M_Z^2-m_\pi^2}\right|^2 = \frac{G_F^2|U_{s4}|^2|U_{si}|^2}{2} + \mathcal{O}\left(\frac{m_\pi^2}{M_Z^2}\right). \]
%%
%which produces the standard expression:
%%
%\[  \Gamma_i = \frac{1}{8\pi}|U_{s4}|^2|U_{si}|^2 G_F^2 f^2_\pi m_\pi m_\mu^2\left(1-\frac{m^2_{\pi^0}}{m^2_s}\right)^2. \]
%%
and therefore
%
\[  \Gamma =  \left(1-\sum_{\alpha \neq s}  |U_{\alpha 4}|^2\right) \sum_{\alpha \neq s} |U_{\alpha 4}|^2 \frac{1}{8\pi} G_F^2 f^2_\pi m_\pi m_\mu^2\left(1-\frac{m^2_{\pi^0}}{m^2_s}\right)^2. \]

\subsection{$N\to \nu e^+e^-$}

We have three diagrams with different mediators (see \reffig{fig:NCdiag} and \reffig{fig:CCdiag}); however, they turn out to have very similar matrix elements and can be combined before square-summing. 

It helps to introduce the following Mandelstam/Dalitz variables, 
%
\[ t=\left(p_1-q_1\right)^2,\qquad u=\left(p_1-q_3\right)^2 \qquad\text{and}\qquad s=\left(p_1-q_2\right)^2. \]

\subsubsection{NC diagrams}


\begin{figure}[t!]
\centering
\begin{fmffile}{ster_decay_1}
	\begin{fmfgraph*}(130,110)
		\fmfstraight
		\fmfleftn{i}{9}
		\fmfrightn{o}{9}
		\fmf{fermion, tension=1.0}{i3,v1,o6}
		\fmf{boson, tension=2.0, label=$Z^\prime$}{v1,v2}
		\fmf{fermion}{o4,v2,o2}
		\fmf{phantom,tension=1.2}{i2,v1}
		\fmflabel{$\nu_4$}{i3}
		\fmflabel{$\nu_i$}{o6}
		\fmflabel{$e^-$}{o2}
		\fmflabel{$e^+$}{o4}
		\marrow{a}{up}{top}{$p_1$}{i3,v1}
		\marrow{b}{up}{top}{$q_1$}{v1,o6}
		\marrow{c}{up}{top}{$q_2$}{v2,o4}
		\marrow{d}{down}{bot}{$q_3$}{v2,o2}
	\end{fmfgraph*}
\end{fmffile}
%
\hspace{1cm}
%
\begin{fmffile}{ster_decay_2}
	\begin{fmfgraph*}(130,110)
		\fmfstraight
		\fmfleftn{i}{9}
		\fmfrightn{o}{9}
		\fmf{fermion, tension=1.0}{i3,v1,o6}
		\fmf{boson, tension=2.0, label=$Z$}{v1,v2}
		\fmf{fermion}{o4,v2,o2}
		\fmf{phantom,tension=1.2}{i2,v1}
		\fmflabel{$\nu_4$}{i3}
		\fmflabel{$\nu_i$}{o6}
		\fmflabel{$e^-$}{o2}
		\fmflabel{$e^+$}{o4}
		\marrow{a}{up}{top}{$p_1$}{i3,v1}
		\marrow{b}{up}{top}{$q_1$}{v1,o6}
		\marrow{c}{up}{top}{$q_2$}{v2,o4}
		\marrow{d}{down}{bot}{$q_3$}{v2,o2}
	\end{fmfgraph*}
\end{fmffile}
%

\caption{\label{fig:NCdiag}The two NC Feynman diagrams for tree-level $N$ decay into a lepton-antilepton pair of the same flavour.}

\end{figure}

The NC diagrams are shown in \reffig{fig:NCdiag}. We compute the $Z^\prime$ mediated diagram first. Its matrix element reads
%
\begin{align*}
%
i\mathcal{M} = -\left(\frac{g}{\sqrt{2}c_W}\right)^2\frac{d^{4i}}{(p_1-q_1)^2-\mu^2} \overline{u}(q_1)\gamma^\mu P_\text{L} u(p_1) \overline{u}(q_3)\gamma_\mu\left(d^e_V-d^e_A\gamma^5\right)
v(q_2),  
%
\end{align*}
%
where $\mu$ is the mass of the $Z^\prime$. 
%
We can write this as
%
\[ \mathcal{M} = \overline{u}(q_1)\gamma^\mu P_\text{L} u(p_1) \overline{u}(q_3)\gamma_\mu\left(V+A\gamma^5\right)
v(q_2), \] 
%
where
%
\[  V = -\left(\frac{g}{\sqrt{2}c_W}\right)^2\frac{d^{4i}d^e_V}{t-\mu^2}\qquad\text{and}\qquad A = +\left(\frac{g}{\sqrt{2}c_W}\right)^2\frac{d^{4i}d^e_A}{t-\mu^2}. \]

The matrix element for the $Z$-mediated diagram is identical with the coupling constants and mediator mass swapped,
%
\[  V = -\left(\frac{g}{\sqrt{2}c_W}\right)^2\frac{c^{4i}c^e_V}{t-M_Z^2}\qquad\text{and}\qquad A = +\left(\frac{g}{\sqrt{2}c_W}\right)^2\frac{c^{4i}c^e_A}{t-M_Z^2}. \]


\subsubsection{CC diagram}

\begin{figure}[t!]
\centering
\begin{fmffile}{ster_decay_3}
	\begin{fmfgraph*}(130,110)
		\fmfstraight
		\fmfleftn{i}{9}
		\fmfrightn{o}{9}
		\fmf{fermion, tension=1.0}{i3,v1,o6}
		\fmf{boson, tension=2.0, label=$W$}{v1,v2}
		\fmf{fermion}{o4,v2,o2}
		\fmf{phantom,tension=1.2}{i2,v1}
		\fmflabel{$\nu_4$}{i3}
		\fmflabel{$e^-$}{o6}
		\fmflabel{$\nu_i$}{o2}
		\fmflabel{$e^+$}{o4}
		\marrow{a}{up}{top}{$p_1$}{i3,v1}
		\marrow{b}{up}{top}{$q_3$}{v1,o6}
		\marrow{c}{up}{top}{$q_2$}{v2,o4}
		\marrow{d}{down}{bot}{$q_1$}{v2,o2}
	\end{fmfgraph*}
\end{fmffile}

\caption{\label{fig:CCdiag}The CC Feynman diagram for tree-level $N$ decay into a lepton-antilepton pair of the same flavour.}

\end{figure}

Although this diagram looks different, we can actually manhandle it into the same form 
as the NC diagrams. The matrix element is given by
%
\begin{align*}
%
i\mathcal{M} &= -\left(\frac{g}{\sqrt{2}}\right)^2\frac{U^*_{e4}U_{ei}}{(p_1-q_3)^2-M_W^2} \overline{u}(q_3)\gamma^\mu
P_\text{L} u(p_1) \overline{u}(q_1)\gamma_\mu P_\text{L}
v(q_2), \\ 
%
&= +\left(\frac{g}{\sqrt{2}}\right)^2\frac{U^*_{e4}U_{ei}}{(p_1-q_3)^2-M_W^2} \overline{u}(q_1)\gamma^\mu
P_\text{L} u(p_1) \overline{u}(q_3)\gamma_\mu P_\text{L} v(q_2),
%
\end{align*}
%
where we have used a Fierz identity\footnote{Nice reference: 0412245} in the last line.
%
This means we can rewrite it in the form of the other diagrams:
%
\begin{align*}
%
\mathcal{M} &= \overline{u}(q_3)\gamma^\mu P_\text{L} u(p_1) \overline{u}(q_1)\gamma_\mu\left(V+ A\right)v(q_2),
%
\end{align*}
%
where we have defined:
%
\begin{align*}
%
V = +\frac{g^2}{4}\frac{U^*_{e4}U_{ei}}{u-M_W^2} \qquad\text{and}\qquad A = - \frac{g^2}{4}\frac{U^*_{e4}U_{ei}}{u-M_W^2}.
%
\end{align*}
%
%For reference, $u = m^2_s(1-y)$ and $\gamma = m_s^2/(M_W^2-m_s^2)$; therefore,
%
%\begin{align*}
%
%g_V(y) = \frac{1}{2}\frac{\gamma}{m_s^2}\frac{g_\nu g_e}{1 + \gamma y} \qquad\text{and}\qquad g_V(y) =- \frac{1}{2}\frac{\gamma}{m_s^2}\frac{g_\nu g_e}{1 + \gamma y} .
%%
%\end{align*}
%

\subsubsection{All the diagrams!}

All the matrix elements have identical Dirac structures, and we can group all
three together in terms of generalised chiral couplings. These have kinematic
dependence because of the propagators, but this will only be a complication
when we are trying to integrate over the phase space at the end.
%

We define the new couplings, 
%
\begin{align*}
%
g_V(t,u) &=  -\left(\frac{g}{\sqrt{2}c_W}\right)^2\frac{d^{4i}d^e_V}{t-\mu^2} -\left(\frac{g}{\sqrt{2}c_W}\right)^2\frac{c^{4i}c^e_V}{t-M_Z^2} +\frac{g^2}{4}\frac{U^*_{e4}U_{ei}}{u-M_W^2},\\ 
%
g_A(t,u) &=  +\left(\frac{g}{\sqrt{2}c_W}\right)^2\frac{d^{4i}d^e_A}{t-\mu^2} +\left(\frac{g}{\sqrt{2}c_W}\right)^2\frac{c^{4i}c^e_A}{t-M_Z^2} -\frac{g^2}{4}\frac{U^*_{e4}U_{ei}}{u-M_W^2}. 
%
\end{align*}
%

Using these definitions, the full matrix element is given by, 
%
\begin{align*}
%
i\mathcal{M} = \overline{u}(q_1)\gamma^\mu P_\text{L} u(p_1)
\overline{u}(q_3)\gamma_\mu\left[g_V(t,u)+g_A(t,u)\gamma^5\right] v(q_2),  
%
\end{align*}
%
but we'll drop the arguments of the couplings for convenience (we'll need them
again on integration). On squared-averaging, we find
%
\[ \overline{\left|\mathcal{M}\right|^2} =  A^{\mu\nu}B_{\mu\nu},\]
%
where
%
\begin{align*}
%
A^{\mu\nu} &\equiv \text{tr}\left[\slashed{q_1}\gamma^\mu
P_\text{L}\left(\slashed{p_1}+m_s\right)\gamma^\nu P_\text{L} \right],\\
B^{\mu\nu} &\equiv  \text{tr}\left[\left(\slashed{q_3}+m_3\right)\gamma^\mu
\left(g_V^*+g_A^*\gamma^5\right)\left(\slashed{q_2}-m_2\right)\gamma^\nu
\left(g_V+g_A\gamma^5\right) \right].
%
\end{align*}
%
Standard fiddling leaves us with, 
%
\begin{align*}
%
A^{\mu\nu} &\equiv 2\left[q_1^\mu p_1^\nu + q_1^\nu p_1^\mu - (p_1\cdot
q_1)g^{\mu\nu} + i\varepsilon^{\alpha\mu\beta\nu}\left(q_1\right)_\alpha
\left(p_1\right)_\beta\right],\\
%
B^{\mu\nu} &\equiv  4\left(|g_V|^2+|g_A|^2\right)\left[q_3^\mu q_2^\nu +
q_3^\nu q_2^\mu - (q_2\cdot q_3)g^{\mu\nu} \right]\\ 
&\qquad+
i8\left(\frac{g^*_Vg_A+g_Vg^*_A}{2}\right)\varepsilon^{\alpha\mu\beta\nu}(q_3)_\alpha(q_2)_\beta - 4m_2m_3\left(|g_V|^2-|g_A|^2\right)g^{\mu\nu},
%
\end{align*}
%
and therefore, 
%
\begin{align*} 
%
A^{\mu\nu}B_{\mu\nu} &= 16\left[ \left|g_V+g_A\right|^2\left(q_1\cdot
q_2\right) \left(p_1\cdot q_3\right) + \left|g_V-g_A\right|^2\left(q_1\cdot
q_3\right) \left(p_1\cdot q_2\right)\right.\\ 
&\qquad\qquad \left.+ m_2m_3\left(|g_V|^2-|g_A|^2\right)\left(p_1\cdot q_1\right)\right]. 
%
\end{align*}
%
In our Mandelstam-esque variables, 
%
\begin{align*} (p_1\cdot q_2) = \frac{m_s^2 + m_2^2 - s}{2},\qquad
(p_1\cdot q_3) &= \frac{m_s^2 + m_3^2 - u}{2} , \qquad \left(p_1\cdot q_1\right) = \frac{m_s^2+m_1^2 - t}{2}, \\ 
%
(q_1\cdot q_3) = \frac{s -m_1^2 - m_3^2}{2},\qquad (q_1\cdot q_2) &= \frac{u - m_1^2 - m_2^2}{2},\qquad (q_2\cdot q_3) = \frac{t - m_2^2 - m_3^2}{2}.
%
\end{align*}
%
We therefore have, 
%
\begin{align*} \overline{\left|\mathcal{M}\right|^2} &= A^{\mu\nu}B_{\mu\nu},\\  
%
&= 16\left[ \left|g_V+g_A\right|^2\left(q_1\cdot q_2\right) \left(p_1\cdot
q_3\right) + \left|g_V-g_A\right|^2\left(q_1\cdot q_3\right) \left(p_1\cdot
q_2\right) + m_2m_3\left(|g_V|^2-|g_A|^2\right)\left(p_1\cdot q_1\right)  \right],\\
%
&=4\left[ \left|g_V+g_A\right|^2 (u - m_1^2 - m_2^2)(m_s^2 + m_3^2 - u)\right.\\
%
&\qquad\qquad \left.+ \left|g_V-g_A\right|^2 (s -m_1^2 - m_3^2)(m_s^2 + m_2^2 - s) + \left(|g_V|^2-|g_A|^2\right)\frac{m_2m_3}{2}\left( m_s^2+m_1^2 - t\right) \right].
%
\end{align*}
%

\subsubsection{Dealing with the PMNS}

This diagram actually occurs for any outgoing mass state, and we need to sum
over these diagrams incoherently (at the matrix element squared level).
However, it is easier if we do this summation by hand, as we can then express
the answer in terms of the normal $U_{\alpha 4}$ elements (and we minimize the 
number of times we perform the numerical integrals that will be necessary for
the total rate).

%
Defining some convenient functions, 
%
\begin{align*}  f(u) &= 4(u - m_1^2 - m_2^2)(m_s^2 + m_3^2 - u),\\  
g(s) &= 4(s -m_1^2 - m_3^2)(m_s^2 + m_2^2 - s),\\
h(s,u) &= 2m_2m_3\left(s + u - m_2^2 - m_3^2\right), \end{align*}
%
we can rewrite the matrix element as,
%
\begin{align*} \overline{\left|\mathcal{M}\right|^2} &=\left|g_V+g_A\right|^2 f(u)
+ \left|g_V-g_A\right|^2g(s) + \left(|g_V|^2-|g_A|^2\right)h(s,u).
%
\end{align*}
%

With all the matrix elements included, we can express the matrix element summed over all outgoing mass states as 
%
\begin{align*} 
%
\overline{\left|\mathcal{M}\right|^2} &= \left(1-\sum_{\alpha\neq s}|U_{\alpha 4}|^2\right)\left( \sum_{\alpha\neq s}|U_{\alpha 4}|^2\left[f_1(s,t,u)+\delta_{e\alpha}f_3(s,t,u)\right]\right) + \left(1-|U_{e 4}|^2\right)|U_{e 4}|^2f_2(s,t,u).
%
\end{align*}
%
with the definitions,
%
\begin{align*}  
%
f_1(s,t,u) &= \left( \frac{A}{t-\mu^2} + \frac{B}{t-M_Z^2}\right)\left(\frac{Af(u) + Dh(s,u)}{t-\mu^2} + \frac{Bf(u)+Eh(s,u)}{t-M_Z^2}\right),\\
%
f_2(s,t,u) &= \left( \frac{F}{u-M_W^2} \right)^2h(s,u),\\
%
f_3(s,t,u) &= \left( \frac{F}{u-M_W^2} \right)\left(\frac{Ah(s,u) + 2Dg(s)}{t-\mu^2} + \frac{Bh(s,u)+2Eg(s)}{t-M_Z^2}\right),
%
\end{align*}
%
and we keep going,
%
\begin{align*}  
%
A = 2\sqrt{2}\left(\frac{M_Z}{m_s}\right)^2G_F\alpha \frac{d^{4i}}{U^*_{s4}U_{si}}(d^e_V-d^e_A), \\
B = 2\sqrt{2}\left(\frac{M_Z}{m_s}\right)^2G_F\beta \frac{c^{4i}}{U^*_{s4}U_{si}}(c^e_V-c^e_A), \\
D = 2\sqrt{2}\left(\frac{M_Z}{m_s}\right)^2G_F\alpha\frac{d^{4i}}{U^*_{s4}U_{si}}(d^e_V+d^e_A), \\
E = 2\sqrt{2}\left(\frac{M_Z}{m_s}\right)^2G_F\beta \frac{c^{4i}}{U^*_{s4}U_{si}}(c^e_V+c^e_A), \\
F = - 2\sqrt{2}\left(\frac{M_Z}{m_s}\right)^2G_F\gamma c_W^2. 
%
\end{align*}


\subsubsection{Total rate}

The decay rate is given by
%
\begin{align*} 
%
d\Gamma &= \frac{1}{(2\pi)^5}\frac{1}{64m^3_s}\overline{\left|\mathcal{M}\right|^2}
du dt d^2\Omega d\phi.
%
\end{align*}

To find the total rate we integrate the matrix element up. The angular integrals just add prefactors, but the Mandelstam integrals require a bit of care (for later use we consider non-zero $m_1$, but recall that our matrix element assumes $m_1=0$ so we have to impose that before doing the integrals),
%
%\[  \Gamma = \frac{1}{(2\pi)^3}\frac{1}{32m^3_s}\int_{(m_1+m_2)^2}^{(m_s-m_3)^2}du\int_{t_-}^{t_+}dt\,\overline{\left|\mathcal{M}\right|^2}. \]
%%
%For fixed $s$ the limits on the $t$ integral are: 
%%
%\[ t_\pm = \left(E_2^* + E_3^*\right)^2 - \left(|p_2^*| \pm |p_3^*|\right)^2, \]
%%
%where $\left(E^*_i\right)^2 - \left(p^*_i\right)^2 = m_i^2$ and 
%%
%\[   |p^*_2|^2 = \frac{\lambda(u,m_1^2,m_2^2)}{4u} \qquad\text{and}\qquad  |p^*_3|^2 = \frac{\lambda(u,m_s^2,m_3^2)}{4u}. \]
%%
%
%Or, 
%
\[  \Gamma = \frac{1}{(2\pi)^3}\frac{1}{32m^3_s}\int_{(m_2+m_3)^2}^{(m_s-m_1)^2}dt\int_{u_+}^{u_-}du\,\overline{\left|\mathcal{M}\right|^2}. \]
%
\[ u_\pm = \left(E_1^* + E_2^*\right)^2 - \left(|p_1^*| \pm |p_2^*|\right)^2, \]
%
where $\left(E^*_i\right)^2 - \left(p^*_i\right)^2 = m_i^2$ and 
%
\[   |p^*_1|^2 = \frac{\lambda(m_s^2,t,m_1^2)}{4t} \qquad\text{and}\qquad  |p^*_2|^2 = \frac{\lambda(t,m_2^2,m_3^2)}{4t}. \]
%

These can of course be done numerically, however, doing this repeatedly turned
out to be intolerably slow. I'll go a bit further analytically, albeit
at the (further) expense of elegance.

First we expand the SM propagators, assuming the scales of the process are well
below the SM weak boson masses. After this, we see that the functions $f_i$ are
made up for fairly simple elementary integrals, all of which can be either
precomputed or expressed as special functions. The first set of elementary
integrals are
%
\begin{align*}  I_{mp} &\equiv \int^{m_s^2}_{4m_e^2} dt \int_{u_+}^{u_-} du
\frac{t^m}{(t-\mu^2)^p},\\ &=
\frac{\left(m_s^2-4m_e^2\right)^\frac{5}{2}\left(4m_e^2\right)^{m-\frac{1}{2}}}{\left(4m_e^2-\mu^2\right)^p}\frac{4}{15}F_1\left(\frac{3}{2},\frac{1}{2}-m,p,\frac{7}{2},-\alpha,\beta\right).
\end{align*}
%
where $\alpha = \frac{m_s^2-4m_e^2}{4m_e^2} $, $\beta =\frac{m_s^2-4m_e^2}{\mu^2-4m_e^2} $ and $F_1(a,b1,b2,c,x,y)$ is the Appell F1 hypergeometric function, which has some nice series 
respresentations (and can be analytically continued to a much larger domain), but enters here through its intergral representation
%
\[  F_1(a,b1,b2,c,x,y) = \frac{\Gamma\left(c\right)}{\Gamma\left(a\right)\Gamma\left(c-a\right)}\int_0^1 z^a(1-z)^{c-a-1}(1-xz)^{-b1} (1-yz)^{-b2}dz. \]
%
We also need to define another set of integrals, 
%
\[   J^{mn}_{p} \equiv \int_{4m_e^2}^{m_s^2}dt \int_{u_+}^{u_-}du \frac{t^m
u^n}{(t-\mu^2)^p}. \]
%
I didn't manage to find a general formulae for these integrals, but we will
only need $n=1$ and $n=2$. These can be expressed as
%
\begin{align*}
%
J^{m1}_p =
&\frac{\left(m_s^2-4m_e^2\right)^\frac{7}{2}\left(4m_e^2\right)^{m-\frac{1}{2}}}{\left(4m_e^2-\mu^2\right)^p}\frac{8}{105}F_1\left(\frac{3}{2},\frac{1}{2}-m,p,\frac{9}{2},-\alpha,\beta\right)\\ 
%
&+
\frac{\left(m_s^2-4m_e^2\right)^\frac{5}{2}\left(4m_e^2\right)^{m+\frac{1}{2}}}{\left(4m_e^2-\mu^2\right)^p}\frac{1}{15}F_1\left(\frac{3}{2},\frac{1}{2}-m,p,\frac{7}{2},-\alpha,\beta\right).
%
\end{align*}
%
Of the $n=2$ series, we only need $J^{02}_p$ which can be computed 
%
\begin{align*} J^{02}_p =
&\frac{\left(m_s^2-4m_e^2\right)^\frac{5}{2}\left(4m_e^2\right)^{-\frac{1}{2}}}{\left(4m_e^2-\mu^2\right)^p}
\frac{4}{45}\sum_{q=-1}^2\left(4m_e^2\right)^q d_q
F_1\left(\frac{3}{2},\frac{1}{2}-q,p,\frac{7}{2},-\alpha,\beta\right).
\end{align*}
%
where the coefficients $d_q$ are given by 
%
\begin{align*} d_{-1} = -m_e^2m_s^4,\qquad d_{0} = 3m_e^4 + 5m_e^2m_s^2 +
m_s^4,\qquad d_{1} = -2(2m_e^2 +m_s^2),\qquad d_{2} = 1.  \end{align*}
%
I find that the following decomposition holds for the integral of the first
function,
%
\begin{align*}
%
\int_{4m_e^2}^{m_s^2}dt \int_{u_+}^{u_-} du\,f_1(s,t,u) &= -4A^2 I_{22} +
\left[4A^2(m_s^2+2m_e^2) - 2ADm_e^2\right]I_{12}\\
%
~~~~& + \left[ -4A^2m_e^2(m_s^2+m_e^2) + 2ADm_e^2m_s^2\right]I_{02} +
\frac{8AB}{M_Z^2}I_{21}\\
%
~~~~& + \left[ -\frac{8AB}{M_Z^2} (m_s^2+2m_e^2) + \frac{2m_e^2}{M_Z^2} (AE +
BD) \right]I_{11} \\
%
~~~~& + \left[ \frac{8AB}{M_Z^2}m_e^2(m_s^2+m_e^2) -
\frac{2(AE+BD)}{M_Z^2}m_e^2m_s^2 \right]I_{01} - \frac{4B^2}{M_Z^4}I_{20}\\
%
~~~~& + \left[ \frac{4B^2}{M_Z^4} (m_s^2+2m_e^2) -
\frac{2BEme^2}{M_Z^4}\right]I_{10} \\
%
~~~~& + \left[ -\frac{4B^2}{M_Z^4}m_e^2(m_s^2+m_e^2) +
\frac{2BE}{M_Z^4}m_e^2m_s^2 \right]I_{00}. 
%
\end{align*}
%
For the second function,
%
\begin{align*}
%
\int_{4m_e^2}^{m_s^2}dt \int_{u_+}^{u_-} du\,f_2(s,t,u) &=
-\frac{2F^2m_e^2}{M_W^4}I_{10} + \frac{2F^2m_e^2m_s^2}{M_W^4}I_{00}
%
\end{align*}
%
Believe it or not, the third function is even uglier, 
%
\begin{align*}
%
\int_{4m_e^2}^{m_s^2}dt \int_{u_+}^{u_-} du\,f_3(s,t,u) &=
\left[2FB\frac{m_e^2m_s^2}{M_W^2M_Z^2} -
8FE\frac{m_e^2(m_s^2+m_e^2)}{M_W^2M_Z^2}\right]I_{00}\\
%
~~~~&+ \left[-2FB\frac{m_e^2}{M_W^2M_Z^2} +
8FE\frac{m_s^2+2m_e^2}{M_W^2M_Z^2}\right]I_{10} - \frac{8FE}{M_W^2M_Z^2}I_{20}
\\
%
~~~~& + \left[-2AF\frac{m_e^2m_s^2}{M_W^2} -
8FD\frac{m_e^2(m_s^2+m_e^2)}{M_W^2}\right]I_{01}\\
%
~~~~&+ \left[2AF\frac{m_e^2}{M_W^2} -
8FD\frac{m_s^2+2m_e^2}{M_W^2}\right]I_{11} + \frac{8FD}{M_W^2}I_{21}\\
%
~~~~& + \frac{8FE(2m_e^2+m_s^2)}{M_W^2M_Z^2}J^{01}_0 -
\frac{8FE}{M_W^2M_Z^2}J^{02}_0 + \frac{16FE}{M_W^2M_Z^2}J^{11}_0 \\
%
~~~~& - \frac{8FD(2m_e^2+m_s^2)}{M_W^2}J^{01}_1 + \frac{8FD}{M_W^2}J^{02}_1 +
\frac{16FD}{M_W^2}J^{11}_1. 
%
\end{align*}



\subsection{$N \to \nu \mu^+\mu^-$  (TO DO)}

This would also be enhanced, just as for the electron case. The calculation is
identical with masses and mixing elements swapped. Apart from these factors the
couplings are universal. I have not coded it up or checked it. In fact, as there is 
no reason for it being suppressed, so it might be a reason to keep the sterile
mass under the dimuon threshold.


\appendix
\newpage
\section{\label{app:vertices}Vertices}

For reference, here we give our notation for all the vertex factors. Every
vertex in our model is identical to the SM apart from the neutrino and neutral
current interactions:
%
\begin{align*}  
%
-\mathcal{L}_\text{I} \supset &\sum_{f\in\{e,u,d,\dots\}}
\overline{f}\gamma^\mu \left[
\frac{g}{2c_W}\left(c^f_V-c^f_A\gamma^5\right)Z_\mu  +
\frac{g}{2c_W}\left(d^f_V-d^f_A\gamma^5\right)Z^\prime_\mu \right]f, \\
%
&+\overline{\nu}_i\gamma^\mu \left[
\frac{g}{2c_W}c^{ij}\left(1-\gamma^5\right)Z_\mu  +
\frac{g}{2c_W}d^{ij}\left(1-\gamma^5\right)Z^\prime_\mu \right] \nu_j +
\overline{\nu_i}\gamma^\mu f^{i\alpha}W^+_\mu P_\text{L} e_\alpha. 
%
\end{align*}
%
where the undefined constants are given for the neutrinos by
%
\begin{align*}
%
f^{i\alpha} &= \frac{g}{\sqrt{2}}U^*_{\alpha i}, \\
%
c^{ij} &= \left(\delta_{ij} - U^*_{si}U_{sj}\right)\frac{c_\beta - s_\beta s_W
t_\chi}{2} - U^*_{si}U_{sj}Q_s \frac{g_X}{g}\frac{c_W s_\beta}{c_\chi}, \\
d^{ij} &= \left(\delta_{ij} - U^*_{si}U_{sj}\right)\frac{c_\beta s_W t_\chi +
s_\beta}{2} + U^*_{si}U_{sj}Q_s \frac{g_X}{g}\frac{c_W c_\beta}{c_\chi}.
\end{align*}
%
And we'll need the following charged fermion couplings,
%
\begin{align*} c^e_V = c_\beta \left(2s_W^2-\frac{1}{2}\right) -
\frac{3}{2}s_\beta s_W t_\chi, \qquad&\qquad c^e_A = -\frac{c_\beta - s_\beta
s_W t_\chi}{2}, \\
%
c^u_V = c_\beta \left(\frac{1}{2}-\frac{4}{3}s_W^2\right) + \frac{5}{6}s_\beta
s_W t_\chi,\qquad&\qquad c^u_A = \frac{c_\beta + s_\beta s_W t_\chi}{2}, \\
%
c^d_V = c_\beta \left(-\frac{1}{2}+\frac{2}{3}s_W^2\right) - \frac{1}{6}s_\beta
s_W t_\chi, \qquad&\qquad c^d_A = -\frac{c_\beta + s_\beta s_W t_\chi}{2}, \\
%
d^e_V = \frac{3}{2}c_\beta s_Wt_\chi + s_\beta\left(-\frac{1}{2} -
2s^2_W\right), \qquad&\qquad d^e_A = -\frac{s_\beta + c_\beta s_W t_\chi}{2},
\\
%
d^u_V = -\frac{5}{6}c_\beta s_Wt_\chi + s_\beta\left(\frac{1}{2} -
\frac{4}{3}s^2_W\right),\qquad&\qquad d^u_A = \frac{s_\beta + c_\beta s_W
t_\chi}{2}, \\
%
d^d_V = \frac{1}{6}c_\beta s_Wt_\chi + s_\beta\left(-\frac{1}{2} +
\frac{2}{3}s^2_W\right),\qquad & \qquad d^d_A = -\frac{s_\beta + c_\beta s_W
t_\chi}{2}.
%
\end{align*}
%
Throughout the preceding expressions $c_\theta = \cos\theta$, $s_\theta =
\sin\theta$ and $t_\theta = s_\theta/c_\theta$. The angle $W$ denotes the
Weinberg angle, $\chi$ is the kinetic mixing parameter and $\beta$ is defined
by
%
\[   \tan(2\beta) = \frac{s_W \sin(2\chi)}{\left(\frac{\mu}{v}\right)^2 -
\cos(2\chi) - c_W^2 s_\chi^2}, \]
%
with $\mu$ being the (tree-level) mass of the $Z^\prime$ boson and $v$ the
Higgs VEV.

%%%%%%%%%%%%%%%%%%%%%%%%%%%%%%%%
%%%%%%%%%%%%%%%%%%%%%%%%%%%%%%%%
%%%%%%%%%%%%%%%%%%%%%%%%%%%%%%%%

\bibliographystyle{apsrev4-1}
\bibliography{ZP}{}

\end{document}

